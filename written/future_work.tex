\section{Future Work}
% current status of project
% aim1
% - only thin missing is validation.
% aim2
% - marginal model done
% - ambiguous data work in progress
% - complex modeling of frameshifts missing (?)
% - biological frameshifts missing
% aim3(\& 4)
% - nothing done for aim3 (\& 4)
%%%%%%%%%%%%%%%%%%%%%%%%%%%%%%%%%%%%%%%%%%%%%%%%%%%%%%%%%%%%%%%%%%%%%%%%%%%%%%%%
% add features to coati-alignpair?
% I can speed up pairwise aln via SIMD
% add more features?
%
%  MSA. If aim 4 is included, then skip sampling aln space.
%  After sampling aln space, improve initial msa? Joint estimation of guide
%  tree? Uncertainty? Anything missing on pairwise? Other indel models?
%
\green{Not convinced about this section.}

Looking forward, the logical and most beneficial next step for COATi should be
extending the current model into a multiple sequence aligner (MSA).
The first addition would be an algorithm that can assemble an initial alignment
both given a phylogenetic tree and building a guide tree when not available.
An iterative refinement step would follow by sampling alignment space in search
of better alternatives.
This would transform COATi into a complete and widely used tool.