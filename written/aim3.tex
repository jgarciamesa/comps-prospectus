\subsection{Aim 3 - Estimate parameter values for COATi's model}

% \2 Background \& Motivation
% 	\3 Applying a model without biologically meaningful parameters is not
% 		optimal.
% 	\3 Recap of chapter 1?


The development of new models and tools that help understand natural phenomena
moves science forward.
COATi will help alleviate the expensive data curation steps that cause large
amounts of information to be discarded, thus improving sequence alignment and
the vast array of downstream analyses that follow.
% coati can align sequences from any taxa
In addition, the model is designed to properly handle a wide variety of
molecular data including pseudo-genes, with an emphasis on protein coding
sequences.

% while coati will have a positive impact, for the best possible results, its'
% important to be able to use accurate parameter values for each dataset.
While COATi will have a positive impact in the field, developing feature-rich
models can present users with a challenge if left alone to tune its parameters.
Thus, COATi will be capable of inferring biologically meaningful parameter
estimates from sequence data.

% \2 EM description
\subsubsection{Expectation-Maximization algorithm}
% double check this description
The expectation-maximization algorithm (EM)
\parencite{dempster_laird_rubin_EM_1977} is a classic method for deriving
maximum likelihood estimates (MLE) of parameters in statistical models with
latent variables.
% a bit more description? expectation \& maximization steps?
%   iterative method
This iterative algorithm alternates between an expectation step and a
maximization step until a convergence threshold is achieved.
During the expectation step, information about the hidden data is inferred,
which is then used to improve parameter estimates in the maximization step.
% it is suitable for our case. "has proven valuable" ??
The efficacy of the EM algorithm has been proven in the context of molecular
evolution (e.g.\cite{holmes2002expectation,holmes_em_2005}).
% Add more information? A one-line paragraph looks weird?
Therefore, I will use an EM approach to infer parameter values estimates from
sequence data for COATi.

% \2 Estimation of substitution parameters
% \subsubsection{Model parameters}
\subsubsection{Model parameters: substitution parameters}
% 	\3 Underlying DNA model and stationary DNA frequencies.
%       \4 Equations ?
% Underlying DNA evolutionary model GTR, upon which codon substitution model is
%  built.
% Describe (briefly) GTR model
% Likelihood for GTR ?
COATi offers the possibility to use a custom substitution model by providing a
substitution matrix.
In addition, the built-in options are MG94 and ECM models.
While both models characterize the codon to codon interactions, ECM specifies
the codon frequencies while MG94 does not.
For the latter, an underlying DNA substitution model is required.
COATi will feature the popular general time reversible model (GTR)
\parencite{tavare_gtr_1986} when using MG94.
% point out that by "using" GTR, many other "sub-models" can also be specified
% (from JC, to HKY or TN93).
A characteristic of GTR is its ability to encode other well-known DNA models
that can be seen as sub-cases such as JC69 \parencite{jukes1969mammalian},
HKY \parencite{hasegawa1985dating}, or TN93 \parencite{tamura1993estimation}.
GTR is composed of ten total parameters, four nucleotide frequencies $\pi_i$
and six transition rate parameters $\sigma_j$.
% something about marginal model? Does that change the number of parameters?
%  changes number of UNDERLYING parameters but not user parameters
In addition, one parameter, coefficient of selection $\omega$, is required for
constructing MG94.

% Parameters for the \red{marginal} substitution model implemented in COATi are
% four stationary nucleotide frequencies $\pi_i$, six transition rates for the GTR
% model $\sigma_j$, and coefficient of selection $\omega$.

% 	\3 Rest of MG94 parameters.
%       \4 Equations ?
% MG94 (brief) description

% \2 Estimation of indel parameters
\subsubsection{Model parameters: indel parameters}
% brief description of indel model
% likelihood/log-likelihood
% description of parameters
% comment about possibility to \green{play around} with more complex models,
%  i.e. g -> i and d ("split" gap into insertion and deletion)

% reference to coati's simplified model (in DP section, aim1?)
The indel model, as described in figure \ref{fig:evolution-fst}-b,
distinguishes between insertion and deletions, with two governing parameters,
gap opening $g$ and gap extension $e$.
The probability that a gap occurs follows a geometric distribution with
parameter $g$.
% while the length of each gap is, in turn, modeled using a
% geometric distribution with parameter $e$.
%
% The indel \green{portion} of the likelihood function for the model in COATi is
% composed of a gap opening ($g$) part and a gap extending ($e$) part
%
% \begin{equation}
%     \mathcal{L} = \mathcal{L}_g \cdot \mathcal{L}_e
% \end{equation}
%
% each section defined
%
% \begin{equation}
%     \mathcal{L}_g = (1-g)^{2S+M+N-T} \cdot g^G
% \end{equation}
% \begin{equation}
%     \mathcal{L}_e = (1-e)^{G} \cdot e^{T-G}
% \end{equation}
%
% where $G$ is the expected number of gaps, $S$ the expected number of
% substitutions, $T$ is the total length of gaps, and $M$ and $N$ represent the
% length for each unaligned sequence.
The model can be extended by splitting both parameters to be event specific,
i.e. insertion opening $i_o$, insertion extension $i_e$, deletion opening $d_o$,
and deletion extension $d_e$.

% Given the transition probabilities in figure \ref{fig:dp-model}, the likelihood
% equation for the indel model is
%
% one last end transition?
% \begin{equation}
%     \mathcal{L} = [(1-g)^2 \cdot s]^{T_1} \cdot g^{T_2} \cdot
%     [(1-g) \cdot g]^{T_3} \cdot [(1-e) \cdot (1-g)]^{T_4} \cdot e^{T_5} \cdot
%     [(1-e) \cdot g]^{T_6} \cdot [1-e]^{T_7} \cdot e^{T_8}
% \end{equation}
%
% where $s$ is the probability of substitution, $g$ is the gap opening weight,
% $e$ is the gap extension weight, and $T_*$ are the number of transitions of each
% type
%
% \begin{align*}
% T_1:& M \rightarrow M\\
% T_2:& M \rightarrow I\\
% T_3:& M \rightarrow D\\
% T_4:& I \rightarrow M\\
% T_5:& I \rightarrow I\\
% T_6:& I \rightarrow D\\
% T_7:& D \rightarrow M\\
% T_8:& D \rightarrow D
% \end{align*}

% 	\3 Power-law models of indel lengths?
% talk about possibility of \green{playing around} with more complex models,
%  e.g. power-law model

% \2 Develop EM
% \subsubsection{Develop EM}

% Given the model(s) describe previously, I will develop an EM for estimating
% parameter values for COATi given sequencing data, thus optimizing the results.
% The E-step will infer values for the green{hidden/unobserved} parameters of the
% model, i.e. ??

% 	\3 Simulated data.
% how to pick parameters/ parameter sets
% what program to use for simulation (DAWG?)
% how to evaluate/measure error, convergence, etc.
  % initial values for EM must cover homogeneously the sample space to avoid
  % getting stuck in local optima.
% Ziqi's suggestion: talk more about validation/simulation process.
% 	\3 Real data.

\subsubsection{Validation}

A common approach for validation is to generate data with a wide set of known
parameters values and assert that the estimates are correct.
I will use DAWG \parencite{cartwright2005dawg}, an open-source C++ sequence
evolution simulator able to generate sequence alignments.
% only codon model is Goldman & Yang, but I should be able to add MG94
DAWG is well suited to generate a dataset for testing given its ability to
specify both a substitution model (e.g. MG94) and an indel model.
As in COATi, DAWG allows gaps to happen anywhere in the sequence, including
within codons, and to span any number of bases, thus allowing frameshifts.

%% talk about local maxima/global maxima & sample space for initial guess?
% A vulnerability of the EM algorithm is that it can converge \green{at/to} a local maxima
% \green{without a warning}.
% Therefore, to ensure the global maxima is found, considering multiple initial guesses
% that \green{cover} homogeneously the sample space is \green{important}.

%% talk about number of sequences, lengths, pipeline, parameter values, etc?

%% validation with real dataset?
% Following \textit{in silico} validation, it's common to test a real/biological
% dataset ??
