%%%%%%%%%%%%%%%%%%%%%%%%%%%%%%%%%%%%%%%%%%%%%%%%%%%%%%%%%%%%%%%%%%%%%%%%%%%%%%%%
\begin{abstract}     											           %
%%%%%%%%%%%%%%%%%%%%%%%%%%%%%%%%%%%%%%%%%%%%%%%%%%%%%%%%%%%%%%%%%%%%%%%%%%%%%%%%

Unpublished reference genomes tend to have artifacts that, if not corrected, can
impact downstream analysis including phylogenetic inference, ancestral sequence
reconstruction, and gene annotation.
Within coding sequences, common artifacts include abiological frameshifts and
early stop codons.
While these are eventually fixed for model organisms, for many species this is
not the case, requiring curation efforts that discard large amounts of data.
Current aligners depend primarily on amino acid translations, generally only
support in-frame indels that occur between codons, and are not robust to
artifacts.
Here I discuss the development of a new statistical sequence alignment software
that will be robust to artifacts in unpolished genomes, incorporate codon
models, and support complex indels.


% Sequence alignment is an essential method in bioinformatics and the basis of
% many analyses including phylogenetic inference, ancestral sequence reconstruction,
% and gene annotation.
% Errors made in alignment reconstruction can impact downstream analyses leading
% to erroneous conclusions in comparative and functional genomics.
% While detailed models of codon evolution are used in phylogenetics, alignment
% reconstruction still depends primarily on amino-acid translations.
% While some alignment software supports codon models, these are not typically
% used and are not robust to artifacts.
% More importantly, current software only supports in-frame indels that occur
% between codons, thus not optimally aligning indels that occur between codons.

% To address this need, I am developing COATi, a new statistical sequence aligner
% that will incorporate a robust codon model while supporting complex indel models.
% This will allow users to reduce the amount of discarded data while generating
% more accurate sequence alignments.
% COATi will be feature rich and combine modern statistical models of molecular
% evolution.
\end{abstract}
