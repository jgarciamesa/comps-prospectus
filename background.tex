\section{Background}
% why is the problem I'm working on interesting?

% figure with number of artifacts & early stop codons found in real data?
%   similar to Reed's figure but create my own, maybe ask for scripts/ideas.

% information age?
Advancements in sequencing technology and the increasing affordability of new
equipment has generated an overflow of genomic information.
The abundance of data being processed today is orders of magnitude greater than
two decades ago \green{with no slow down in sigh}.
As a result, computational tools are required to keep up with such pace to
properly handle large projects.

The available deluge of genomic data is not free of artifacts,
(\red{Add reference or preliminary data figure about artifacts})
which requires curation practices that discard large amounts of information in
hopes to improve the signal to noise ratio.
Uncorrected errors in genomic datasets can lead to erroneous functional and
comparative genomic \green{studies} \red{(citation?)}.

% To address this \green{need}, I plan on developing COATi, a multiple sequence
% aligner (MSA) that will be able to correctly handle artifacts present in
% heterogeneous genomic datasets.
% \green{The tool will include robust codon models of DNA substitution, fine
% calibration of parameters, and user friendly operability.}

% Having a \green{suitable adjectives for excellent} aligner is not enough if
% researchers are left alone to infer biologically meaningful parameters for
% COATi's model.
% Therefore, a \green{key feature} will be the ability to \green{derive} parameter
% estimates from data for an accurate result.

% \begin{itemize}
% 	\item From proteins to alignment, including CDS, codons, mutations (
% 		substitutions and indels), and reading frame.
% 	\item Current aligners and their flaws/weaknesses.
% 	\item Indel figure.
% 	\item Biological advancements if coati is implemented.
% \end{itemize}

% Proteins are molecules that perform many critical roles in the body, including
% \green{examples}.
% Proteins are formed by one or more chains of amino acids, and each amino acid
% sequence is defined by a section of a gene's DNA known as the coding sequence
% (CDS).
% A CDS is formed by a sequence of three nucleotide units known as codons, each of
% which is translated into an amino acid.
% Mutations that affect CDSs include substitutions, which change nucleotides, and
% insertions and deletions (indels) which add or delete nucleotides.
% \red{TODO: finish paragraph and modify so that it's not so similar to coati's
% grant}.
%
% Indels of length not multiple of three produce a frameshift in all downstream
% codons, resulting in a change in their reading frame and therefore a completely
% different protein translation.
% Indels are categorized according to their position in a codon or phase.
% Phase-0 indels take place between codons whereas phase-1 and phase-2 indels
% take place inside codons.

% alignment paragraph

Sequence alignment is considered a fundamental task in bioinformatics and a
cornerstone step in comparative and functional genomic studies
\parencite{sequence_alignment_rosenberg_2009}.
An alignment is a hypothesis of which characters from two or more sequences are
homologous \parencite{problems_cartwright_2009}.
Inference of homology is an inherently difficult problem since the objective is
backtracking unique and unobservable historical events
\parencite{sequence_aln_morrison_2010}.
Sequence alignment is also essential to other analyses, including the
identification of conserved motifs, estimation of evolutionary divergence
between sequences, inference of phylogenetic relationships
\parencite{MSA_kumar_2007}, identification of disease-associated mutations,
measurement of selection, among others
\parencite{sequence_alignment_rosenberg_2009}.

% about programs still not fully addressing sequence homology
Modern sequence analysis began with the heuristic homology algorithms of
Needleman and Wunsch in 1970 \parencite{identification_smith_1981} and has
progressed to arrive at current aligners such as BAli-Phy(?), CLUSTALW, MAFFT,
MACSE, PRANK \red{add citations}.
However, the alignment of molecular sequences is, in practice, often seen as a
tool and the alignment inference as an ad hoc problem
\parencite{morrison_MSA_2018}.

% what current aligners are missing
Despite the existence of models for codon evolution, a common strategy in the
alignment of coding sequences is \green{to do so based on amino acid
translations \red{citation}}.
While this approach offers several improvements over DNA models, it discards
information, fails in the presence of artifacts, and has been shown to
underperform compared to alignment at the codon level.
\green{
Although some aligners incorporate codon substitution models (e.g. BAli-Phy,
PRANK, MACSE), they do not support frameshifts.}
% lack of statistical model?
While indels are rarely modeled to appear within codons, it has been estimated
that this is often the case \parencite{indel_phases_zhu_2019}.

\red{indel figure.}

Current methods are not able to keep up with the amount of information
generated.
Frameshifts are common in coding-sequence datasets \red{citation}.
However, these are expected to be errors due to strong purifying selection.
Identifying canonical coding sequences to patch this issue is the most
\green{approachable/accessible} solution and yet often unsuccessful.
Improving the annotation quality or re-sequencing with higher quality entails
high costs with little reward.

Therefore, researchers are ill-equipped to deal with uncurated heterogeneous
datasets.
To address this need, I propose to develop COATi, a tool that will be able to
generate sequence alignments while correcting for artifacts in a feature-rich
and user-friendly \green{package}.

Having a \green{suitable adjectives for excellent} aligner is not enough if
researchers are left alone to infer biologically meaningful parameters for
COATi's model.
Therefore, a \green{key feature} will be the ability to \green{derive} parameter
estimates from data for an accurate result.
