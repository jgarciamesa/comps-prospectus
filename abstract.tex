%%%%%%%%%%%%%%%%%%%%%%%%%%%%%%%%%%%%%%%%%%%%%%%%%%%%%%%%%%%%%%%%%%%%%%%%%%%%%%%%
\section*{Project Summary} %1 page											   %
%%%%%%%%%%%%%%%%%%%%%%%%%%%%%%%%%%%%%%%%%%%%%%%%%%%%%%%%%%%%%%%%%%%%%%%%%%%%%%%%

% OVERVIEW %%%%%%%%%%%%%%%%%%%%%%%%%%%%%%%%%%%%%%%%%%%%%%%%%%%%%%%%%%%%%%%%%%%%%
%  Describe the activity that would result if the proposal were funded and state
% the objectives and methods to be employed.
\textbf{Overview}\\
\begin{itemize}
    \item Explosion of (genomics) data, drowning on data.
    \item Artifacts are common and cause data to be filtered out.
    \item Uncorrected errors in genomic datasets lead to bad alignments, etc.
    \item "This project will develop new computational methods to" solve those
        challenges.
    \item And teach computational skills to researchers and students. (?)
\end{itemize}
%% Intro paragraph
% Hook: Explosion of genomic data has opened a lot of possibilities but also
%       brought along challenges.

% What is known: current knowledge in the field.
%

% More info?

% Gap in knowledge
%       Artifacts are still an issue that increases curation and discards data.

% Critical need
%       Errors in genomic datasets lead to [...] errors in downstream analyses.

%%
% Hook:
\blue{ Sequence alignment is a fundamental task in bioinformatics and a
cornerstone step in comparative and functional genomic studies.
% What is known: current knowledge in the field.
% Important topic, many comp tools exist with a variety of different approaches.
Due to its relevance, extensive investments have been made into developing the
variety of multiple sequence aligners that exist today.
% More info? Maybe something about it being an ad hoc problem
%
% Gap in knowledge
Despite these efforts, alignments produced by current computational tools
are suboptimal and often modified by hand before publication \parencite{hand_curation_morrison_2009}
or manually verified before accepted into benchmark databases.
% Critical need
Consequently, there is an critical need for improving sequence alignment accuracy.
}

%%%%%%%%%%%%%%%%%%%%%%%%%%%%%%%%%%%%%%%%%%%%%%%%%%%%%%%%%%%%%%%%%%%%%%%%%%%%%%%%
% Intellectual Merit %%%%%%%%%%%%%%%%%%%%%%%%%%%%%%%%%%%%%%%%%%%%%%%%%%%%%%%%%%%
%  Describe the potential of the proposed activity to advance knowledge within
% its own field or across different fields, including the qualifications of the
% team to conduct the project and the extent to which the proposed activities
% suggest and explore creative, original, or potentially transformative concepts.
\textbf{Intellectual Merit:}\\
\begin{itemize}
    \item COATi, "MSA that uses advance models to capture the biological and
        technical processes that shape genomic data."
    \item "Cutting edge models of protein-coding evolution".
    \item Handle heterogeneous datasets with "high and low quality data".
    \item User friendly and open-source.
    \item "Improve the analysis of genomic sequences, etc."
\end{itemize}

%% Second paragraph
% long-term goal
Therefore, I propose to develop new computational methods % that will keep up
% with current demands (speed, quantity of data, user friendly/usability).

% proposal objectives
This will be done by \red{developing} COATi, a tool capable of generating multiple
sequence alignments that corrects artifacts/errors in genomic data.

% rationale
%   Utilizing "models of protein coding-sequence evolution". COATi will be able
%   to handle heterogeneous datasets/ high and low-quality data.

% pay-off
%   This will reduce data curation efforts, provide more information by discarding
%   less data, and thus improve (downstream analyses).

%%
% long-term goal
\blue{Therefore I propose to develop a novel approach that will generate more accurate multiple
sequence alignments (MSAs) improving downstream analyses including phylogenetic inference, detection
of selection and co-evolution, and identification of disease associated mutations.
% Proposal objectives
This will be done by developing a machine learning model that bridges the gap between the
output of state-of-the-art aligners and the improved results of manual curation by experienced
biologists and bioinformaticians.
% Rationale
Thus, utilizing robust and well-known machine learning models to simulate by-hand enhancements
is expected to produce high quality results,
% Pay-off
consequently allowing a time-saving and reproducible MSA pipeline.
}

% Broader Impacts %%%%%%%%%%%%%%%%%%%%%%%%%%%%%%%%%%%%%%%%%%%%%%%%%%%%%%%%%%%%%%
%  Describe the potential of the proposed activity to benefit society and
% contribute to the achievement of specific, desired societal outcomes.
\textbf{Broader Impacts:}\\
\begin{itemize}
    \item Software developed will be valuable in "numerous projects in comparative
        and functional genomics".
    \item This will have a positive impact in many fields.
    \item Ability to use larger datasets without discarding such amount of
        information and "reducing artifacts".
    \item Teach Carpentry workshops for "instructors at regional, tribal, and
        community colleges."
\end{itemize}
