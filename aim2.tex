\subsection{Aim 2 - Artifacts in Genomic Datasets}

% \2 Background \& Motivation
Errors and artifacts are a common problem in genomic datasets, notably
frameshifts and early stop codons.
In order to prevent errors from leading to inaccurate downstream analyses,
current practices involve time and resource-consuming curation efforts that
discard large amounts of data, consequently losing information.

Genomes for model organisms are often of high-quality after being refined over
many iterations and having their coding sequences meticulously curated.
On the contrary, non-model organisms typically have lower-quality genomes that
have been only partially curated.
Low-quality genomes often lack the amount of sequencing data needed to fix
artifacts, including missing exons, erroneous mutations, and indels
\parencite{jackman2018tigmint}.

FSTs and their well established methods provide an efficient framework to
statistically align sequences from a non-model organisms against sequences from
a model organism.
Therefore, I plan to equip COATi to correctly handle artifacts present in
heterogeneous genomic datasets.
COATi-alignpair will be able to model the alignment of a low-quality coding
sequence against a high-quality reference as a path through two FSTs: evolution
FST and an FST that models the error-causing processes of sequencing and
assembly.

% \2 Marginal substitution model
\subsubsection{Marginal substitution model}

%  justify why we chose that model and what are the benefits/biological implications?
% default model will be marginal MG94
\green{Explanation is weird but can't find a better approach. Let me know what
you think :)}

Codon substitution models define the rate of change between nucleotide triplets,
with the implicit assumption that codons from both sequences are accurately
sequenced and mapped.
To lessen this assumption and leverage the alignment on the high-quality
reference sequence, over the low-quality sequence, the default codon model
implemented on the substitution FST will be a marginalized codon model.
Given the substitution matrix $P$ defined in aim 1, the marginalized version is
defined

\[ P'_{ij} = P(j | i, \text{pos}) \]

Conceptually, $P'_{ij}$ represents the probability that codon $i$ from the
ancestor sequence (high-quality) changes to nucleotide $j$ of the descendant
(low-quality) sequence at position pos $\in \{0,1,2\}$ of the reading frame.
This model emphasizes the position of a nucleotide substitution in a codon
(phase) and is not affected by erroneous adjacent nucleotides while adding more
weight to the high-quality sequence.

% \red{marginal substitution FST figure?}
% \caption{Marginal substitution FST. This FST is created from a 64x3x4 matrix,
%          replacing the 64x64 earlier version in the evolution FST.}

% \2 Artifacts and ambiguous data
\subsubsection{Artifacts and ambiguous data}

% \red{TODO(?): analyze existing DB for frameshifts and early stop codons. }
% 	\3 Analyze existing databases and plot distribution of artifacts
% 		(frameshifts \& early stop codons) and ambiguous nucleotides
% 		(has this been done before?).

% would it make sense to analyze existing DB maybe even alignment benchmarks
%  used for validating alignments in search for artifacts/amb nucs?

% 	\3 Introduce IUPAC ambiguous nucleotide support into marginal COATi
% 		model.
% 		\4 Theory and implementation for descendant sequence.
% 		\4 Theory for reference sequence, implementation subject to time.
In the DNA alignment problem, the alphabet of nucleotides is ideally composed
of four residues \{A,C,G,T\} plus gap \{-\}.
Unfortunately, errors in sequencing and assembly introduce uncertainty that is
represented by ambiguous residues.
To represent all possibilities, the alphabet can be extended to include up to
sixteen symbols, according to standardize IUPAC notation
\parencite{cornish_1985_nomenclature}.

% \red{table with IUPAC symbols?} - probably not necessary

% 	\3 How to treat ambiguous nucleotides.
% 		\4 Average vs most probable nucleotide (“best”).
Given that sequences from model organisms have been polished and refined over
time, it is reasonable to assume that the high-quality sequence in our model to
be free of ambiguous nucleotides.
In addition, adding support for all IUPAC nucleotide symbols for the reference
sequence would add complexity to the marginal substitution model without a
promise of a clear payoff.
However, I plan on exploring the possibility of adding this feature.

In contrast, low-quality sequences are expected to contain ambiguous nucleotides
and COATi will be equipped to handle them.
A common strategy to handle ambiguous nucleotides, when not directly removing the
containing codon, is to average over all possibilities.
However, an ambiguous residue represents a single nucleotide that was
inaccurately interpreted instead of an average of possibilities.
To my knowledge, no alternative approaches have been explored for handling
uncertain nucleotides in alignment.
Therefore, I plan on evaluating other strategies to treat ambiguous nucleotides,
such as selecting the nucleotide that best replaces an ambiguous base.

% \2 Model sequencing error
% \subsubsection{Model sequencing error}
\subsubsection{Model frameshifts}

% 	\3 Sequencing FST
% 		\4 Gap unit size: 1 vs 3 vs 1\&3
% 		\4 Incorporate base-calling error profile?

% \textbf{Sequencing FST.}
% base-calling error profile? "M to S generate matches; however, here they can
%  introduce single-nucleotide errors, which can generate stop-codon artifacts"

% gaps 1 vs 3 vs 1 \& 3
% Errors in sequencing and assembly are often expected to be the cause of
% frameshifts present in comparative genomic datasets \red{citation}.
Indel FST (\ref{fig:evolution-fst}-b) models the insertions and deletions when
aligning a pair of sequences, including frameshift-causing indels, by allowing
gaps of any length to occur at any position.
% This transducer is governed by two parameters: gap opening and gap extension.
To distinguish between frameshift-causing indels and indels that do not disrupt
the reading frame, a more parameter-rich transducer can be designed.
When setting the indel FST to only allow gaps of length multiple of three (one
or more codons), this can be composed with a similar transducer that only
allows gaps of length one or two.
With this approach longer frameshifts can be modeled by combining an indel
(length multiple of three) with a frameshift (length one or two).
I will compare the performance of the initial indel FST with the
higher-parameter model that specifically models frameshifts.

% Sequencing FST is a transducer that specifically models frameshifts, allowing
% only gaps of length one or two to occur.
% When setting the indel FST to only allow gaps of length multiple of three (one
% or more codons), both transducers can be composed to distinguish between
% frameshift-causing indels and indels that do not disrupt the reading frame.
% With this approach larger frameshifts can be modeled by combining a frameshift
% (gap of length one or two) with an indel (gap of length multiple of three).
% I will compare the performance of the initial indel FST that allows gaps of any
% length with the higher-parameter model that specifically models frameshifts.

% \red{Sequencing FST figure.}

Assuming frameshifts are false positives, COATi will provide the option to
correct frameshifts by adding ambiguous nucleotides that restore the original
reading frame.
This will ensure the alignment produced by our tool is properly formatted to use
by any software in comparative genomic pipelines.

% \2 Biological frameshifts
\subsubsection{Biological frameshifts}

While most frameshifts found in the alignment of protein coding sequences are
expected to be errors due to due to strong purifying selection, in some cases
frameshifts are believed to be biological \parencite{hu2012predicting}.
To my knowledge, this particular case is not addressed by any current aligners,
therefore, I plan on developing an approach that can model biological
frameshifts.
% \green{More information?}

% \subsubsection{Validation}
